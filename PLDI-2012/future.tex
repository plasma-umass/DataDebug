In future work, we plan to explore applying data debugging to other
domains, including database management systems and scientific
computing environments, especially R. Both domains will require
tailoring of the existing algorithms to their particular context. For
databases, we plan to treat as computations both stored procedures as
well as cached queries. While it is straightforward to apply data
debugging to databases when the queries have no side effects, handling
queries that do modify the database will take some care in order to
avoid an excessive performance penalty due to copying.  Similarly,
data debugging will likely need to take into account features of the R
language in order to work effectively. Finally, we are interested in
exploring the effectiveness of data debugging in conventional programming
language settings.

We also plan to explore other ways of ranking impacts. Our current
approach looks for outliers assuming a normal distribution; this
approach is conservative, as Section~\ref{sec:analysis} explains. We
are particularly interested in exploring non-parametric approaches
like kernel density estimators, which can fit arbitrary distributions
and potentially reduce the risk of false positives.
