Testing and static analysis tools can help root out bugs in programs,
but not bugs in data. Checking data for errors is arguably as
important as finding program errors, but lacks effective tool
support. This paper introduces \emph{data debugging}, an approach that
combines program analysis with data analysis to locate data
errors. Since it is impossible to know \emph{a priori} whether data
are erroneous or not, data debugging does the next best thing:
locating data where an error would have the most impact. Data
debugging is particularly promising in the context of data-intensive
programming environments like databases and spreadsheets, which
intertwine data with programs (in the form of queries or formulas).

Data debugging works by building a dependency graph of an entire
computation. It then computes the impact of data on the computation
by evaluating the average effect of replacing it with randomly-selected data
from the same population. This paper presents an implementation of data
debugging in an add-in tool for Microsoft Excel called \checkcell{}.
\checkcell{} highlights values in shades of red proportional to their impact on the
spreadsheet's computation, including charts and formulas. A user study
verifies the effectiveness of using data debugging to find data
errors. \checkcell{} users were able to find errors with XX\%
accuracy, while users without were only able to achieve YY\% accuracy.
