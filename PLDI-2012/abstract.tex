Testing and static analysis tools can help root out bugs in programs,
but not bugs in data. Checking data for errors is arguably as
important as finding program errors, but lacks effective tool
support.
%  Currently, the only approach is manual inspection of
% data. Because inspection is onerous and ineffective at scale, data
% errors are common: for example, error rates in simple manual entry
% tasks are typically XXX\%.
This paper introduces \emph{data debugging}, an approach that combines
program analysis with data analysis to locate input values that have
an unusual effect on the results of computations. These values
simultaneously provide valuable insights into the data and can reveal
errors.  Data debugging is particularly promising in the context of
data-intensive programming environments like databases and
spreadsheets, which intertwine data with programs (in the form of
queries or formulas).

We present a data debugging tool, \checkcell{}, that targets
spreadsheets. \checkcell{} builds a dependency graph of an entire
spreadsheet, including formulas and charts, where the leaves are cells
or ranges of cells. It then computes the influence of every cell by
systematically evaluating the impact of replacing it with any other
item from the same range. Because data errors only matter when they
have a significant impact, \checkcell{} highlights important values in
shades of red proportional to their influence in the spreadsheet.  We
perform a user study to measure the effectiveness of using
\checkcell{} to find injected errors in spreadsheets. \checkcell{} users
were able to find errors with XX\% accuracy, while users without
\checkcell{} were only able to achieve YY\% accuracy.
