Testing and static analysis tools can help root out bugs in programs,
but not bugs in data. Checking data for errors is arguably as
important as finding program errors, but lacks
effective tool support. Currently, the only approach is manual
inspection of data, one at a time. Because inspection is onerous and
ineffective at scale, most data remains full of errors. CITE
SPREADSHEET STUDY OF ERROR RATES.

This paper introduces \emph{data debugging}, an approach that combines
program analysis with data analysis to locate values that have an
unusual effect on the final computation.  These values simultaneously
provide valuable insights into the data and can reveal data
errors.  Data debugging is particularly promising in the context of
data-intensive programming environments where data and programs are
interwined, such as databases (queries and stored procedures) and
spreadsheets (formulas).

We present a data debugging tool, CheckCell, that targets
spreadsheets. CheckCell builds a dependency graph of an entire
spreadsheet, including formulas and charts, where the leaves are cells
or ranges of cells. It then computes the influence of every cell by
systematically evaluating the impact of replacing it with any other
item from the same range. Because data errors only matter when they
have a significant impact, CheckCell highlights important values in
shades of red proportional to their influence in the spreadsheet.  We
perform a user study to measure the effectiveness of using
CheckCell to find injected errors in spreadsheets. CheckCell users
were able to find errors with XX\% accuracy, while users without
CheckCell were only able to achieve YY\% accuracy.
