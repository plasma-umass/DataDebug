Analysis to come.

\paragraph{Normal distribution of impacts.}

Justifies use of trimming (only reporting things that are two standard
deviations)--high confidence that they are outliers.  Average change,
independent (random), same distribution (don't exclude same cell
swap). Use 30 samples.

\paragraph{Asymptotic analysis.}

Cost to construct dependence graph, cost to perform sampling. (Suspend redraw?)

\paragraph{Probability of missing important data.}

30 samples. Odds of missing something important are vanishingly small.
Cite Feller. Limitations: threshold functions.

\paragraph{Limitations}

\begin{itemize}
\item {\bf Array formulas.}
\item {\bf Tables.}
\item {\bf Control flow / HLOOKUP / VLOOKUP.}
\item {\bf Outputs that change data type.}
\item {\bf Macros / side-effects.}
\end{itemize}

\subsection{Impact Outlier Analysis}

Once all impact scores have been computed, only those data whose
impacts cross some threshold of anomalousness should be
reported. \checkcell{} behaves as if the impact scores fit a normal
distribution, and only reports those data whose scores place them more
than two standard deviations away from the mean. In a normal
distribution, that is just under 0.05\% of the population; in other
words, this corresponds to a 95\% confidence level that these are
anomalies.

The use of a normal distribution is conservative, in that it can only
over-report anomalies. This fact derives from two special
characteristics of the normal distribution: its low \emph{skewness}
and \emph{kurtosis}.

The normal distribution has zero skewness, where
skew is the distribution around the mean; in other words, it is
perfectly symmetric. Any non-normal distribution by definition has a
greater number of points either to the left or to the right of the
mean. By choosing outliers from the tails of the normal, \checkcell{}
also includes the skewed tails of any other distribution.

In addition, the normal has either low (3) or zero kurtosis, depending
on the definition of kurtosis, which indicates the heaviness of the
tail. Counting outliers from the perspective of the normal
distribution is also conservative, because it includes all
distributions with heavier tails. Distributions with negative kurtosis
actually have finite \emph{support}, meaning that their tails
eventually become zero, dropping into the x-axis at some point on
either side. It is difficult to argue that these distributions have
any outliers at all, by virtue of the fact that they have extremely
small tails.

% we just need to report outliers in the impact. assuming the impacts
% are normal is a conservative approach: the normal has 0 skewness
% (skew = distribution around the mean -- normal is symmetric, so 0
% skew) and low (either 0 or 3) kurtosis, depending on your definition
% of kurtosis. Every non-normal distribution is by definition more skewed and most
% distributions have a higher kurtosis (heavier tails), so we may
% overmark outliers. We won't find outliers in distributions with
% negative kurtosis, but those are super weird (no tails -- they drop
% below the x-axis at some point on either side), so it's hard to
% argue that they have outliers at all.
