\begin{figure*}[!t]
\centering
\includegraphics[width=5.5in]{execution_time_graph}
  \caption{\checkcell{} execution time. For most of the spreadsheets, \checkcell{} completes its analysis in under 9 seconds; for all but two, it completes in under three minutes (see Section~\ref{sec:execution_time}).\label{fig:execution_time_graph}}
\end{figure*}
 

\begin{table*}[!t]
  \centering \begin{tabular}{l|rrr||r|rrr}
 \small{\bf{Spreadsheet}} & \small{\bf{Formulas}} & \small{\bf{Cells}} & \small{\bf{Cells}} & \small{\bf{Runtime}} & \small{\bf{Dep.}} & \small{\bf{Impact}}   & \small{\bf{Impact}} \\
 & & {\small{\it{raw}}} & {\small{\it{weighted}}} & \small{\it{total (s)}} & \small{\bf{Analysis}} & \small{\bf{Analysis}} & \small{\bf{Scoring}} \\
\hline
\small{3660 schedule S2003} & \small{31} & \small{1} & \small{0} & \small{1.54} & \small{0.73} & \small{0.44} & \small{0.34} \\ 
\small{ReqComp} & \small{54} & \small{162} & \small{0} & \small{1.95} & \small{0.95} & \small{0.52} & \small{0.44} \\ 
\small{Inventory\_Control} & \small{33} & \small{21} & \small{0} & \small{4.71} & \small{1.67} & \small{1.59} & \small{1.42} \\ 
\small{RMRanker95} & \small{79} & \small{54} & \small{11} & \small{7.07} & \small{2.74} & \small{2.38} & \small{1.91} \\ 
\small{Logistikkostnader} & \small{73} & \small{29} & \small{26} & \small{8.88} & \small{3.48} & \small{2.97} & \small{2.40} \\  
\small{HMWK112403} & \small{36} & \small{41} & \small{27} & \small{2.31} & \small{0.88} & \small{0.78} & \small{0.63} \\ 
\small{30day} & \small{125} & \small{92} & \small{30} & \small{3.01} & \small{1.39} & \small{1.31} & \small{0.27} \\ 
\small{2002fairreport} & \small{3} & \small{39} & \small{39} & \small{4.30} & \small{1.29} & \small{1.67} & \small{1.30} \\ 
\small{9620040303160820} & \small{42} & \small{81} & \small{81} & \small{4.77} & \small{1.19} & \small{2.74} & \small{0.81} \\  
\small{Inventory errors} & \small{100} & \small{129} & \small{90} & \small{2.83} & \small{1.25} & \small{1.02} & \small{0.53} \\ 
\small{grades} & \small{227} & \small{661} & \small{96} & \small{154.45} & \small{3.06} & \small{149.85} & \small{1.51} \\ 
\small{expenses\_ans} & \small{57} & \small{60} & \small{120} & \small{3.24} & \small{0.92} & \small{2.15} & \small{0.15} \\ 
\small{grades2002} & \small{61} & \small{143} & \small{123} & \small{2.67} & \small{1.03} & \small{1.11} & \small{0.51} \\ 
\small{csDept-PayrollTimecardEntry} & \small{68} & \small{204} & \small{124} & \small{7.37} & \small{1.85} & \small{4.37} & \small{1.10} \\  
\small{Example\_3} & \small{71} & \small{130} & \small{127} & \small{3.15} & \small{1.22} & \small{1.56} & \small{0.28} \\ 
\small{lmc\_financial} & \small{72} & \small{148} & \small{142} & \small{17.15} & \small{4.69} & \small{7.63} & \small{4.80} \\ 
\small{104r} & \small{22} & \small{146} & \small{144} & \small{6.66} & \small{1.81} & \small{3.26} & \small{1.54} \\ 
\small{TRAIL INVENTORY N\#A850A} & \small{2} & \small{156} & \small{156} & \small{6.15} & \small{1.12} & \small{3.99} & \small{0.99} \\ 
\small{Grades-6\_excerpt} & \small{106} & \small{168} & \small{168} & \small{1.83} & \small{1.10} & \small{0.45} & \small{0.25} \\ 
\small{intresults} & \small{1066} & \small{3158} & \small{239} & \small{318.91} & \small{17.12} & \small{287.63} & \small{14.12} \\ 
\small{OakProducts} & \small{69} & \small{271} & \small{242} & \small{6.82} & \small{1.67} & \small{4.20} & \small{0.91} \\ 
\small{am\_skandia\_fin\_supple\#A80EE} & \small{56} & \small{272} & \small{268} & \small{6.64} & \small{1.53} & \small{4.01} & \small{1.06} \\ 
\small{E04\_AppE\_Census\_Database\_50} & \small{42} & \small{300} & \small{300} & \small{39.04} & \small{4.07} & \small{32.72} & \small{2.22} \\ 
\small{pfi-anxa} & \small{5} & \small{310} & \small{310} & \small{73.56} & \small{16.38} & \small{33.10} & \small{24.05} \\ 
\small{q exhibit54-OEA} & \small{797} & \small{1160} & \small{365} & \small{102.56} & \small{18.03} & \small{68.70} & \small{15.79} \\ 
\small{econ424-fall2003-publ\#A8A23} & \small{93} & \small{517} & \small{384} & \small{62.83} & \small{3.91} & \small{56.96} & \small{1.93} \\ 
\small{Grades\_EEE481\&581} & \small{177} & \small{757} & \small{756} & \small{40.11} & \small{3.31} & \small{35.74} & \small{1.03} \\ 
\small{gpa\_calculator} & \small{80} & \small{80} & \small{819} & \small{115.86} & \small{1.88} & \small{113.67} & \small{0.28} \\ 
\small{s446gradessp04} & \small{335} & \small{1369} & \small{1247} & \small{129.36} & \small{9.76} & \small{113.29} & \small{6.27} \\ 
\small{NEW} & \small{2626} & \small{2574} & \small{2403} & \small{683.32} & \small{115.75} & \small{440.30} & \small{127.23} \\ 
    \end{tabular}%
  \caption{The benchmark suite of 30 spreadsheets, a random sample from the EUSES repository~\cite{Fisher:2005:ESC:1082983.1083242}, ordered by weighted number of cells. The raw number of cells indicates the number of cells that are used in any formula; the weighted number of cells weighs each cell by the number of formulas that depend on it. A breakdown of \checkcell{} execution times (in seconds) appears on the right side.\label{tab:spreadsheet_characteristics}}
\end{table*}

We evaluate \checkcell{} across two dimensions: its execution time,
and its effectiveness at finding actual errors.

Our experimental platform is a 13'' MacBook Air equipped 4GB of RAM
and an Intel Core i5-2557M processor running at 1.70GHz. The operating
system is Windows 7 Professional (SP1), which executes non-virtualized
(via Bootcamp). \checkcell{} was compiled using Microsoft Visual C\#
2010, and runs as an add-in in Microsoft Excel 2010.

\subsection{Execution Time}
\label{sec:execution_time}

To measure the runtime of \checkcell{}, we run it on a random subset
of 30 spreadsheets drawn from the EUSES
corpus~\cite{Fisher:2005:ESC:1082983.1083242}, excluding those that do not contain
formulas.

Table~\ref{tab:spreadsheet_characteristics} includes characteristics
of these spreadsheets, ordered by the number of formulas each
contains. We include two columns that count the number of cells in
different ways. \emph{Cells (raw)} indicates the total number of cells
that participate in any computation. \emph{Cells (weighted)} indicates
the total number of cells, weighted by the number of times each cell
is used in a computation. For example, a cell that is involved in two
computations is counted twice.

\begin{figure*}[!t]
\centering
\includegraphics[width=7in]{intresults_tree}
  \caption{A dataflow graph for the \texttt{intresults} spreadsheet, one of the benchmarks with an unusually high runtime.  The highly-connected clique in the center causes the impact analysis to dominate \checkcell{}'s computation time since a change in a single cell may require the recomputation of a large number of values. \label{fig:intresults_tree}}
\end{figure*}

 
Figure~\ref{fig:execution_time_graph} reports the performance of data
debugging across our spreadsheet suite, ordered by the weighted number
of cells. Table~\ref{tab:spreadsheet_characteristics} includes the full data.

For 19 of the 30 spreadsheets, \checkcell{} takes 9 seconds or less to
complete. Its runtime is less than three minutes for all but two of
the spreadsheets: \texttt{intresults} and \texttt{NEW}, which take 318
seconds and 683 seconds, respectively. The average runtime over all
spreadsheets is 61 seconds; without the two outliers, it is 29
seconds. As our analysis in Section~\ref{sec:asymptotic_analysis}
predicts, the cost of \checkcell{} is generally proportional to the
cost of the impact analysis, which is in turn dependent on the
weighted number of cells.

The spreadsheets that require the most execution time both have by far
the largest number of formulas (1,066 and 2,626), and the latter also
has the largest number of weighted cells (2,403). Their relatively
high execution time is attributable to the fact that cost of impact
analysis increases as the number of formulas increases, since the
Excel recalculation engine must do more work per item tested.

\paragraph{Summary:} For nearly every spreadsheet
 examined, \checkcell{}'s runtime is under three minutes; we believe
 this overhead is acceptable for an error detection tool.

% Info about the benchmarks.

% \subsection{Benchmarks}

%\subsection{Case Studies}

% \paragraph{9-Grades}

\subsection{Error Detection}
\label{sec:user_study}

While \checkcell{} can be used across the EUSES suite, looking for
errors in existing spreadsheets is problematic because we do not know
what the ground truth is. To evaluate \checkcell{}'s efficacy at
finding actual errors, we need errors and ground truth to compare it
against.

Rather than artificially inject errors, we designed an experiment that
allows us to observe real errors produced by people and use
\checkcell{} to find them. We collect human errors by hiring workers
to perform data entry tasks (entering known data) via Amazon's
Mechanical Turk, a popular crowdsourcing platform, and then check
their results with \checkcell{}.

Our ground truth data is drawn from \texttt{3q2000.xls}, a
spreadsheet from the EUSES repository that contains selected financial
information from Fannie Mae. We save the data as a comma-separated
value file (.csv). Mechanical Turk workers were paid 3 cents to
enter 10 of these numerical values at a time into a web form designed to look
like a spreadsheet, shown in Figure~\ref{fig:mturk_task}. To prevent
copying and pasting, we generate an image containing the
comma-separated values. Each worker had the opportunity to perform up
to seven different tasks.

In all, we collected 200 responses from 46 distinct users. Out of
these responses, 14 had omitted data and 52 contained errors, for an
overall error rate of 33\%. The errors can be classified into the following categories:

\begin{itemize}
\item \textbf{Sign omission}, where a negative sign was dropped;
\item \textbf{Magnitude errors}, any change in a value (usually a dropped or spurious digit) that results in an order of magnitude increase or decrease;
\item \textbf{Digit transposition}, where at least two digits are transposed;
\item \textbf{Typos}, any other typographical error (e.g., a mistyped digit).
\end{itemize}

We then inserted the erroneous data back into the spreadsheet one at a
time and ran \checkcell{} to see whether it found any of these
errors. Recall that by design, \checkcell{} reports data with an
unusual impact on any of the calculations. For this
spreadsheet, \checkcell{} always highlights the values in the top row
(the net interest income) because these values have a significant
impact on the spreadsheet; most of the income in this spreadsheet
comes from this row. We classify \checkcell{} as having correctly
found an error if it also highlights an erroneous cell.

For 13 of the 52 erroneous inputs (25\%), \checkcell{} correctly marks
the cell with the error, supporting our hypothesis that locating data
with unusual impact also finds errors. In all but two of these cases,
the error was a magnitude error; such errors are likelier to have an
unusual impact on a computation than all other errors, since they
change the input data dramatically. Even sign omission only causes a
factor of two change in a data element. Nonetheless, 20 of the errors
that \checkcell{} does not report also involve magnitude errors, but
those errors occur in data that do not contribute significantly to any
computation.


\begin{figure*}[!t]
\centering
\includegraphics[width=5.5in]{images/mturk_fuzz_task}
  \caption{The page presented to Mechanical Turk workers to perform data entry tasks in order to collect actual human data entry errors (see Section~\ref{sec:user_study}).\label{fig:mturk_task}}
\end{figure*}


\begin{figure*}[!t]
\centering
\includegraphics[width=5.5in]{images/fannie_mae_outlier}
  \caption{A screenshot of \checkcell{}'s results. In addition to the top row, which has a large impact on the final results, \checkcell{} highlights cell \texttt{G19}, a human data entry error.\label{fig:fannie_mae}}
\end{figure*}

Figure~\ref{fig:fannie_mae} presents a screenshot of \checkcell{}'s
results with one of these errors. In addition to the top
row, \checkcell{} indicates that cell \texttt{G19} has an unusual
impact; this is, in fact, the error. The correct value
for \texttt{G19} is \texttt{-379300000}, and the value entered by the
worker was \texttt{3793000000}: the worker made both a sign error and
an order of magnitude error (one too many 0's).

\paragraph{Summary:} By searching for data with unusual impacts on the spreadsheet, \checkcell{} is able to successfully find actual human data entry errors.

\subsection{Impact Normality}
\label{sec:impact_normal}

\checkcell{} treats inputs whose impact scores are more than two standard deviations from the mean impact as outliers.  When impact scores are normally distributed, we can strongly claim that scores outside 2 standard deviations are unusual, and that standard outlier rejection techniques are non-controversial.  While \checkcell{} does not assume normality for our outlier rejection procedure, we empirically evaluate the distribution of average impacts produced by our suite of benchmarks.

Several of our benchmarks are empty forms and are thus populated only with zero values.  We exclude these spreadsheets from our analysis since the standard deviation of their average impacts is, by definition, zero, and they would thus be vacuously normal.

Our analysis of 23 spreadsheets shows that average impact distributions appear largely normal, thus supporting our normality assumption.  In a standard normal distribution, no more than 5\% of values are found outside 2 standard deviations.  In our evaluation, on average, 1.45\% of impact values fall outside of 2 standard deviations from the mean.  Limiting our analysis to ranges of size 15 or greater, approximately 3.47\% of their impact values fall outside 2 standard deviations.  The latter is a stronger claim, since small distributions tend to have fewer outliers.
