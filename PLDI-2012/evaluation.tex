We evaluate \checkcell{} across two dimensions: its runtime efficiency
and its effectiveness at finding errors.

Our experimental platform is a 13'' MacBook Air equipped 4GB of RAM
and an Intel Core i5-2557M processor running at 1.70GHz. The operating
system is Windows 7 Professional (SP1), which executes non-virtualized
(via Bootcamp). \checkcell{} was compiled using Microsoft Visual C\#
2010, and runs as an add-in in Microsoft Excel 2010.

\subsection{Execution Time}

To measure the runtime of \checkcell{}, we run it on a random subset
of 30 spreadsheets drawn from those spreadsheets in the EUSES
spreadsheet corpus~\cite{Fisher:2005:ESC:1082983.1083242} that contain
formulas.  Table~\ref{tab:spreadsheet_characteristics} includes
characteristics of these spreadsheets, ordered by the number of
formulas each contains. We report the raw number of cells containing
data and the number of these involved in computations; in the latter,
a cell that is involved in two computations is counted twice.

\begin{table}[t!]
  \centering
    \begin{tabular}{l|rrr}
      \textsf{\bf{Spreadsheet}} & \textsf{\bf{Formulas}} & \textsf{\bf{Cells}} & \textsf{\bf{Cells}} \\
                                &                        & \textsf{\bf{(Raw)}} & \textsf{\bf{(Computed)}} \\
    \hline
    \textsf{Foo} & \textsf{1} & \textsf{5}  & \textsf{20} \\
    \textsf{Foo} & \textsf{1} & \textsf{5}  & \textsf{20} \\
    \textsf{Foo} & \textsf{1} & \textsf{5}  & \textsf{20} \\
    \textsf{Foo} & \textsf{1} & \textsf{5}  & \textsf{20} \\
    \textsf{Foo} & \textsf{1} & \textsf{5}  & \textsf{20} \\
    \textsf{Foo} & \textsf{1} & \textsf{5}  & \textsf{20} \\
    \textsf{Foo} & \textsf{1} & \textsf{5}  & \textsf{20} \\
    \textsf{Foo} & \textsf{1} & \textsf{5}  & \textsf{20} \\
    \textsf{Foo} & \textsf{1} & \textsf{5}  & \textsf{20} \\
    \textsf{Foo} & \textsf{1} & \textsf{5}  & \textsf{20} \\
    \textsf{Foo} & \textsf{1} & \textsf{5}  & \textsf{20} \\
    \textsf{Foo} & \textsf{1} & \textsf{5}  & \textsf{20} \\
    \textsf{Foo} & \textsf{1} & \textsf{5}  & \textsf{20} \\
    \textsf{Foo} & \textsf{1} & \textsf{5}  & \textsf{20} \\
    \textsf{Foo} & \textsf{1} & \textsf{5}  & \textsf{20} \\
    \textsf{Foo} & \textsf{1} & \textsf{5}  & \textsf{20} \\
    \textsf{Foo} & \textsf{1} & \textsf{5}  & \textsf{20} \\
    \textsf{Foo} & \textsf{1} & \textsf{5}  & \textsf{20} \\
    \textsf{Foo} & \textsf{1} & \textsf{5}  & \textsf{20} \\
    \textsf{Foo} & \textsf{1} & \textsf{5}  & \textsf{20} \\
    \textsf{Foo} & \textsf{1} & \textsf{5}  & \textsf{20} \\
    \textsf{Foo} & \textsf{1} & \textsf{5}  & \textsf{20} \\
    \textsf{Foo} & \textsf{1} & \textsf{5}  & \textsf{20} \\
    \textsf{Foo} & \textsf{1} & \textsf{5}  & \textsf{20} \\
    \textsf{Foo} & \textsf{1} & \textsf{5}  & \textsf{20} \\
    \textsf{Foo} & \textsf{1} & \textsf{5}  & \textsf{20} \\
    \textsf{Foo} & \textsf{1} & \textsf{5}  & \textsf{20} \\
    \textsf{Foo} & \textsf{1} & \textsf{5}  & \textsf{20} \\
    \textsf{Foo} & \textsf{1} & \textsf{5}  & \textsf{20} \\
    \textsf{Foo} & \textsf{1} & \textsf{5}  & \textsf{20} \\
    \hline
    \end{tabular}%
  \caption{Characteristics of the benchmark suite of spreadsheets evaluated here.\label{tab:spreadsheet_characteristics}}
\end{table}

% Info about the benchmarks.

% \subsection{Benchmarks}

\subsection{Case Studies}

\paragraph{9-Grades}

\subsection{Human-Generated Errors}

